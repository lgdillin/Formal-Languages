% !TEX TS-program = pdflatex
% !TEX encoding = UTF-8 Unicode

% This is a simple template for a LaTeX document using the "article" class.
% See "book", "report", "letter" for other types of document.

\documentclass[20pt]{article} % use larger type; default would be 10pt

\usepackage[utf8]{inputenc} % set input encoding (not needed with XeLaTeX)

%%% Examples of Article customizations
% These packages are optional, depending whether you want the features they provide.
% See the LaTeX Companion or other references for full information.

%%% PAGE DIMENSIONS
\usepackage{geometry} % to change the page dimensions
\geometry{a4paper} % or letterpaper (US) or a5paper or....
% \geometry{margin=2in} % for example, change the margins to 2 inches all round
% \geometry{landscape} % set up the page for landscape
%   read geometry.pdf for detailed page layout information
\usepackage{titlesec}
\usepackage{graphicx} % support the \includegraphics command and options

% \usepackage[parfill]{parskip} % Activate to begin paragraphs with an empty line rather than an indent

%%% PACKAGES
\usepackage{booktabs} % for much better looking tables
\usepackage{array} % for better arrays (eg matrices) in maths
\usepackage{paralist} % very flexible & customisable lists (eg. enumerate/itemize, etc.)
\usepackage{verbatim} % adds environment for commenting out blocks of text & for better verbatim
%\usepackage{subfig} % make it possible to include more than one captioned figure/table in a single float
\usepackage{mathtools}
\usepackage{graphicx} % supports images in latex
% These packages are all incorporated in the memoir class to one degree or another...

\usepackage{graphicx}
\usepackage{subcaption}

%%% Other stuff
\DeclarePairedDelimiter\ceil{\lceil}{\rceil}
\DeclarePairedDelimiter\floor{\lfloor}{\rfloor}

%%% HEADERS & FOOTERS
\usepackage{fancyhdr} % This should be set AFTER setting up the page geometry
\pagestyle{fancy} % options: empty , plain , fancy
\renewcommand{\headrulewidth}{0pt} % customise the layout...
\lhead{}\chead{}\rhead{}
\lfoot{}\cfoot{\thepage}\rfoot{}

%%% SECTION TITLE APPEARANCE
\usepackage{sectsty}
\allsectionsfont{\sffamily\mdseries\upshape} % (See the fntguide.pdf for font help)
% (This matches ConTeXt defaults)

%%% ToC (table of contents) APPEARANCE
\usepackage[nottoc,notlof,notlot]{tocbibind} % Put the bibliography in the ToC
\usepackage[titles,subfigure]{tocloft} % Alter the style of the Table of Contents
\renewcommand{\cftsecfont}{\rmfamily\mdseries\upshape}
\renewcommand{\cftsecpagefont}{\rmfamily\mdseries\upshape} % No bold!

%%% graphics path

\usepackage{listings}
%\begin{lstlisting}[language=java]
%\end{lstlisting}

%%% END Article customizations

%%% nice things to keep around
%\begin{figure}[!htbp]
%  	\centering
%   	\begin{subfigure}[p]{0.5\linewidth}
%    	\includegraphics[width=\linewidth]{}
%   	\end{subfigure}
%\end{figure} 

% \noindent\rule{2cm}{0.4pt} 
%%% puts a small horizontal line

% \mathcal{O} 
%%% big O notation

% \begin{table}[!htbp]
% \caption{Forward slash.}
% \[\begin{array}{c|ccccc} 
% abc/def & 1 & 2 & 3 & 4 & 5\\
% \hline
% 1 & a & b & c & d & e\\
% 2 & f & g & h & i & j\\
% 3 & k & l & m & n & o\\
% \end{array}\]
% \end{table}

%%% The "real" document content comes below...

\title{Formal Languages Final Study guide}
\author{Liam Dillingham}
%\date{} % Activate to display a given date or no date (if empty),
         % otherwise the current date is printed 

\begin{document}
\maketitle

\section{Definitions}
\subsection{Strings}
\begin{itemize}
\item $\Sigma$: alphabet.  An alphabet is a \textit{finite} set of symbols (not including $\epsilon$)
\item $\Sigma^{k}$: strings from the alphabet $\Sigma$ of length $k$
\item $\Sigma^{*}$: The set of all strings over an alphabet (including $\Sigma^{0}$ i.e. $\epsilon$).
\item $\Sigma^{+}$: set of non-empty strings
\item A language $L$ is a set of strings from the alphabet $\Sigma^{*}$ such that $L \subseteq \Sigma^{*}$
\end{itemize}
\subsection{Finite Automata}
\subsubsection{Deterministic Finite Automata}
A \textit{DFA}, labeled as $A$, is defined as $A = (Q, \Sigma, \delta, q_0, F)$, such that:
\begin{enumerate}
\item $Q$: a finite set of \textit{states}
\item $\Sigma$: a finite set of \textit{input symbols}
\item $\delta(q, a)$: a transition function with arguments as $q$: the current state, and $a$: the current input symbol, where $\delta: Q \times \Sigma \rightarrow Q$
\item $q_0$, or the starting state in $Q$
\item $F$: The set of final or accepting states such that $F \subseteq Q$.
\end{enumerate}
\paragraph{Extended Transition Function}  $\hat{\delta}(q,w) = \delta(\hat{\delta}(q,x), a)$\\
The \textit{extended transition function} precisely describes what happens when we start in any state and follow any sequence of inputs i.e. defines $\delta$ for whole words instead of symbols
\paragraph{Language of DFA} if $A$ is a DFA, then $L(A) = \{ w \mid w \in \Sigma^{*}$ and $\hat{\delta}(q_0, w) \in F \}$
\subsubsection{Nondeterministic Finite Automata}
The only difference between a \textit{DFA} and \textit{NFA} is that for an \textit{NFA}, $\delta$ maps to a set of states. that is, $\delta: Q \times \Sigma \rightarrow 2^{Q}$ i.e. $\mathcal{P}(Q)$
\paragraph{Extended Transition Function}
$$\text{basis: } \hat{\delta}(q, \epsilon) = q. \text{  induction: } \hat{\delta}(q,w) = \hat{\delta}(q, xa) = \bigcup_{p \in \hat{\delta}(q,x)}\delta(p,a)$$
\subsubsection{$\epsilon$-Nondeterministic Finite Automata}
For $\epsilon$\textit{-NFA}, we explicitly define transitions for $\epsilon$, i.e. $\delta: Q \times (\Sigma \cup \{ \epsilon \}) \rightarrow 2^{Q}$
\paragraph{Extended Transition Function}
\begin{itemize}
\item \textbf{ECLOSE($q$)}: All the states that $q$ can reach using only $\epsilon$
\item \textbf{ECLOSE($S$)}: $\bigcup_{r \in S}$ \textbf{ECLOSE($r$)}, where $S$ is a set of states
\end{itemize}
For the precise definition, we have: 
$$\text{basis: } \hat{\delta}(q, \epsilon) = \text{\textbf{ECLOSE($q$)}. } \ \hat{\delta}(q,w) = \hat{\delta}(q, xa) = \text{\textbf{ECLOSE}} \Bigg( \bigcup_{p \in \hat{\delta}(q,x)}\delta(p,a) \Bigg)$$
The language described by an $\epsilon$\textit{-NFA}, $A$, is defined as: $A = \{ w \mid \hat{\delta}(q_0, w) \cap F \neq \emptyset \}$.
\subsection{Regular Expressions}
\subsubsection{Operators}
\paragraph{Union:} if $L = \{001,10,11\}$ and $M = \{$
\subsection{Properties of Regular Languages}
\subsubsection{The Pumping Lemma}
\paragraph{The pumping lemma for regular languages}  Let $L$ be a regular language. Then there exists a constant $n$ (which depends on $L$) such that for every string $w$ in $L$ such that $|w| \geq n$, we can break $w$ into three strings, $w = xyz$ such that:
\begin{enumerate}
\item $y \neq \epsilon$
\item $|xy| \leq n$
\item For all $k \geq 0$, the string $xy^{k}z$ is also in $L$
\end{enumerate}
\end{document}







































