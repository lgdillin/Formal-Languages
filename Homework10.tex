% !TEX TS-program = pdflatex
% !TEX encoding = UTF-8 Unicode

% This is a simple template for a LaTeX document using the "article" class.
% See "book", "report", "letter" for other types of document.

\documentclass[20pt]{article} % use larger type; default would be 10pt

\usepackage[utf8]{inputenc} % set input encoding (not needed with XeLaTeX)

%%% Examples of Article customizations
% These packages are optional, depending whether you want the features they provide.
% See the LaTeX Companion or other references for full information.

%%% PAGE DIMENSIONS
\usepackage{geometry} % to change the page dimensions
\geometry{a4paper} % or letterpaper (US) or a5paper or....
% \geometry{margin=2in} % for example, change the margins to 2 inches all round
% \geometry{landscape} % set up the page for landscape
%   read geometry.pdf for detailed page layout information

\usepackage{graphicx} % support the \includegraphics command and options

% \usepackage[parfill]{parskip} % Activate to begin paragraphs with an empty line rather than an indent

%%% PACKAGES
\usepackage{booktabs} % for much better looking tables
\usepackage{array} % for better arrays (eg matrices) in maths
\usepackage{paralist} % very flexible & customisable lists (eg. enumerate/itemize, etc.)
\usepackage{verbatim} % adds environment for commenting out blocks of text & for better verbatim
%\usepackage{subfig} % make it possible to include more than one captioned figure/table in a single float
\usepackage{mathtools}
\usepackage{graphicx} % supports images in latex
% These packages are all incorporated in the memoir class to one degree or another...

\usepackage{graphicx}
\usepackage{subcaption}

%%% Other stuff
\DeclarePairedDelimiter\ceil{\lceil}{\rceil}
\DeclarePairedDelimiter\floor{\lfloor}{\rfloor}

%%% HEADERS & FOOTERS
\usepackage{fancyhdr} % This should be set AFTER setting up the page geometry
\pagestyle{fancy} % options: empty , plain , fancy
\renewcommand{\headrulewidth}{0pt} % customise the layout...
\lhead{}\chead{}\rhead{}
\lfoot{}\cfoot{\thepage}\rfoot{}

%%% SECTION TITLE APPEARANCE
\usepackage{sectsty}
\allsectionsfont{\sffamily\mdseries\upshape} % (See the fntguide.pdf for font help)
% (This matches ConTeXt defaults)

%%% ToC (table of contents) APPEARANCE
\usepackage[nottoc,notlof,notlot]{tocbibind} % Put the bibliography in the ToC
\usepackage[titles,subfigure]{tocloft} % Alter the style of the Table of Contents
\renewcommand{\cftsecfont}{\rmfamily\mdseries\upshape}
\renewcommand{\cftsecpagefont}{\rmfamily\mdseries\upshape} % No bold!

%%% graphics path

\usepackage{listings}
%\begin{lstlisting}[language=java]
%\end{lstlisting}

%%% END Article customizations

%%% nice things to keep around
%\begin{figure}[!htbp]
%  	\centering
%   	\begin{subfigure}[p]{0.5\linewidth}
%    	\includegraphics[width=\linewidth]{}
%   	\end{subfigure}
%\end{figure} 

% \noindent\rule{2cm}{0.4pt} 
%%% puts a small horizontal line

% \mathcal{O} 
%%% big O notation

% \begin{table}[!htbp]
% \caption{Forward slash.}
% \[\begin{array}{c|ccccc} 
% abc/def & 1 & 2 & 3 & 4 & 5\\
% \hline
% 1 & a & b & c & d & e\\
% 2 & f & g & h & i & j\\
% 3 & k & l & m & n & o\\
% \end{array}\]
% \end{table}

%%% The "real" document content comes below...

\title{Formal Languages Homework 10}
\author{Liam Dillingham}
%\date{} % Activate to display a given date or no date (if empty),
         % otherwise the current date is printed 

\begin{document}
\maketitle

\section{Problem 9.1.1}
What strings are:
\subsection{$w_{37}$?}
37 in binary is $100101$. then $1w = 100101$, so $w = 00101$.
\subsection{$w_{100}$?}
100 in binary is $1100100$. then $1w = 1100100$, so $w = 100100$.
\section{Problem 9.2.1}
Show that the halting problem, the set of $(M,w)$ pairs such that $M$ halts (with or without accepting) when given input $w$ is r.e. but not recursive \\
\noindent\rule{2cm}{0.4pt} \\
For a given input $w$, the TM $M$ must decide whether or not the string is in the language. Let $L_u$ be the set of pairs of $(M,w)$ such that $M$ halts on $w$.  By encoding $M$, then we can rename our set $X$ to $L_u$, that is, $L_u(M,w) = \{ (M,w) \ \mid M$ halts on $w \}$.  Then $\overline{L_u}$ is the set of pairs of $(M,w)$ such that $M$ does \textit{not} halt on $w$.  If $L_u$ is recursive, then all pairs of $(M,w)$ are in $L_u$, since a recursive language always halts.  However, $\overline{L_u}$ would have to be recursive as well, since the compliment of a recursive language must also be recursive.  Yet $\overline{L_u}$ is precisely the language of the pairs of $(M,w)$ which do not halt.  Since a recursive language must halt eventually, then determining the set of pairs which halt is recursievely enumerable, since determining which ones do not halt requires infinite computation.
\newpage
\section{Problem 9.2.2}
Using \textit{Ackermann's function}:
\begin{enumerate}
\item $A(0,y)=1$ for any $y \geq 0$
\item $A(1,0)=2$
\item $A(x,0)=x+2$ for $x \geq 2$
\item $A(x+1,y+1)=A(A(x,y+1),y)$ for any $x \geq 0$, and $y \geq 0$
\end{enumerate}
\subsection{a). Evaluate $A(2,1)$}
$A(2,1)=A(1+1, 0+1)$ Rule \#4 $\Rightarrow A(1+1, 0+1) = A(A(1, 0+1), 0)$ \\
$A(1+1, 0+1) = A(A(1, 0+1), 0) = A(A(1,1), 0)$ \\
$A(1,1) = A(0+1, 0+1) = A(A(0, 0+1), 0) = A(A(0,1),0)$ \\
$A(0,1) = 1$. Now substitute \\
$A(1,1) = A(0+1, 0+1) = A(A(0, 1),0) = A(1,0) = 2$. And substituting again, \\
$A(2,1) = A(1+1, 0+1) = A(A(1, 0+1), 0) = A(A(1,1), 0) = A(2, 0) = x+2 = 4$
\subsection{b). What function of $x$ is $A(x,2)$?}
\subsection{c). Evaluate $A(4,3)$}
$A(4,3) = A(3+1,2+1) = A(A(3,2+1), 2)$ \\
$A(3, 2+1) = A(3, 3) = A(2+1, 2+1) = A(A(2, 2+1), 2)$** \\
$A(2, 2+1) = A(2,3) = A(1+1, 2+1) = A(A(1, 2+1), 2)$ \\
$A(1, 2+1) = A(1,3) = A(0+1, 2+1) = A(A(0, 2+1), 2)$*\\
$A(0, 2+1) = A(0,3) = 1$.  Now we subtitute backwards. \\
$A(A(0, 2+1), 2) = A(A(0,3),2) = A(1,2)$. \\
$A(1,2) = A(0+1, 1+1) = A(A(0, 1+1), 1)$ \\
$A(0, 1+1) = A(0,2) = 1$. Subtitute back. \\
$A(1,2) = A(A(0,2), 1) = A(1,1)$. Using results derived from previous questions, $A(1,1) = 2 = A(1,2)$.  Substitute back starting at line *.\\
$A(1,3) = A(A(0, 3), 2) = A(1,2) = 2$ \\
$A(2,3) = A(A(1,3),2) = A(2,2)$ \\
$A(2,2) = A(1+1, 1+1) = A(A(1, 1+1), 1)$ \\
$A(1, 1+1) = A(1,2) = 2$. subtitute \\
$A(2,2) = A(2, 1) = 4$. (From previous question). Now substitute back to **.
$A(3,3) = A(A(2,3), 2).$ Since $A(2,2) = A(2,3) = 4$, then $A(3,3) = A(4, 2)$. \\
$A(4,2) = A(3+1, 1+1) = A(A(3, 1+1), 1)$ \\
$A(3, 1+1) = A(3, 2) = A(2+1, 1+1) = A(A(2, 1+1), 1)$ \\
$A(2, 1+1) = A(2,2) = 4$. substitute \\
$A(3,2) = A(4, 1).$\\

I want to say that at this point I got tired of doing the calculations by hand, so I wrote a python script which computes the Ackermann function.  However, there seemed to be some sort of recursion error with the code at $A(4,3)$ so I computed it for $A(4,1)$ and got 8.

$A(4,1)=8$. Substitute back \\
$A(4,2) = A(A(3,2), 1) = A(8, 1)$. Using our script, we get $A(8,1) =16$. So $A(4,2) = A(8,1) = 16 = A(3,3)$. \\
$A(4,3) = A(16, 2)$ \\
$A(16, 2) =  A(15 + 1, 1 + 1) = A(A(15, 1+1), 1)$

The python code is shown below:
\begin{lstlisting}[language=python]
import sys
sys.setrecursionlimit(8000)

def A(x, y):

    # rule 1
    if x == 0 and y >= 0:
        return 1

    # rule 2
    if x == 1 and y == 0:
        return 2

    # rule 3
    if x >= 2 and y == 0:
        return x + 2

    # rule 4
    if x >= 0 and y >= 0:
        x = x - 1
        y = y - 1

        return A(A(x, y + 1), y)

    raise Exception('Something terrible happened!')

print(A(16,2))
\end{lstlisting}

\end{document}
